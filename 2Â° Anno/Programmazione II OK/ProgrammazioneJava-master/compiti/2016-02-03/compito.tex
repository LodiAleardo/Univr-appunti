\documentclass{article}[10pt]
\usepackage[pdftex]{graphicx}
\usepackage{amsfonts}
\usepackage[italian]{babel}
\usepackage{graphicx}

%****************enlarge layout
\textheight     243.5mm
\topmargin      -20.0mm
\textwidth      480pt
\hoffset        -80pt
%*****************theorems and such
\newcounter{esnu}
\newenvironment{esercizio}{\medskip \noindent {\bf Esercizio\addtocounter{esnu}{1} \arabic{esnu}}}{}
\pagestyle{empty}
\newcommand{\liff}{\mathrel{\leftrightarrow}}   % Logical IFF Symbol
\newcommand{\metaiff}{\Longleftrightarrow}      %iff in metatheory

\begin{document}

%\begin{tabular}{llclcr}
% \hspace{-35pt} &{\bf COGNOME:} & \hspace{100pt}        &{\bf NOME:}    & \hspace{100pt}        &{\bf MATRICOLA:}%\hspace{35pt} \\
%\hline
%\end{tabular}
\begin{center} {\bf Esame di Programmazione II, 3 febbraio 2016}\end{center}
%\`

Si consideri un'enumerazione di tutti i partiti che si presentano alle elezioni:

{\small\begin{verbatim}
public enum Partito {
  PENSIONATI,
  FELICE,
  FLOREALE,
  CAOTICO,
  BASSOTTI;

  public final static int NUMERO_PARTITI = elementi().length;

  public static Partito[] elementi() {
    return values();
  }

  public int indice() {
    for (int pos = 0; pos < NUMERO_PARTITI; pos++)
      if (elementi()[pos] == this)
        return pos;

    throw new RuntimeException("impossibile");
  }
}
\end{verbatim}
}

\noindent
Si noti che questa enumerazione definisce una costante per il numero
totale dei partiti, un metodo \texttt{elementi()} che permette di ottenere l'array
di tutti i partiti e un metodo \texttt{indice()} che restituisce l'indice di un dato partito
dentro l'array di tutti i partiti.

\begin{esercizio}
\textbf{[11 punti]}
Si completi la classe che rappresenta un'elezione. Essa permette di registrare i voti per
i vari partiti:

{\small\begin{verbatim}
public class Elezioni {
  ...
  public void vota(Partito partito) { ...registra un altro voto per il partito indicato }

  @Override
  public String toString() {
    ...sulla base dei voti registrati per ciascun partito, restituisce una stringa del tipo:

    Risultati elezioni:

    1      PENSIONATI:  7762 voti (33.03%)
    2          FELICE:  3216 voti (13.68%)
    3        FLOREALE:  5695 voti (24.23%)
    4         CAOTICO:  5731 voti (24.39%)
    5        BASSOTTI:  1098 voti (4.67%)    
  }

  protected final int numeroVotiEspressi() {
    ...restituisce il numero totale di voti registrati
  }

  protected final int votiPer(Partito partito) {
    ...restituisce il numero totale di voti registrati per il partito indicato
  }
}
\end{verbatim}}

\end{esercizio}

\begin{esercizio}
  \textbf{[10 punti]}
  Si definisca una sottoclasse \texttt{ElezioniIstogramma} di
  \texttt{Elezioni}, la cui unica differenza rispetto alla sua superclasse
  \`e nel metodo \texttt{toString()} che, \emph{oltre a generare la stessa stringa
  che nella superclasse}, deve aggiungere un istogramma di 50 caratteri
  che indicano in propozione i voti ottenuti per ciascun partito. Per esempio
  deve restituire una stringa del tipo:

  {\small\begin{verbatim}
    Risultati elezioni:

1      PENSIONATI:  7762 voti (33.03%)
2          FELICE:  3216 voti (13.68%)
3        FLOREALE:  5695 voti (24.23%)
4         CAOTICO:  5731 voti (24.39%)
5        BASSOTTI:  1098 voti (4.67%)

111111111111111122222233333333333344444444444455
  \end{verbatim}}

\end{esercizio}

\begin{esercizio}
  \textbf{[10 punti, facoltativo per chi ha gi\`a fatto il progetto degli anni passati]}
  Si consideri la seguente classe che rappresenta una coppia di un partito e dei voti
  che ha ottenuto a un'elezione:

  {\small\begin{verbatim}
public class VotiPerPartito {
  private final Partito partito;
  private final int voti;

  public VotiPerPartito(Partito partito, int voti) {
    this.partito = partito;
    this.voti = voti;
  }

  @Override
  public String toString() {
    return partito + ": " + voti;
  }
}
  \end{verbatim}}

  \noindent
  Scrivete le modifiche al codice fin qui realizzato in modo da rendere gli oggetti
  di classe \texttt{Elezioni} ed \texttt{ElezioniIstogramma} iterabili su
  \texttt{VotiPerPartito} (questo significa che iterando
  su tali oggetti si deve ottenere oggetti \texttt{VotiPerPartito}
  che rappresentano i voti per ciascun partito, uno dopo l'altro).
  
\end{esercizio}

\end{document}
