\documentclass{article}[10pt]
\usepackage[pdftex]{graphicx}
\usepackage{amsfonts}
\usepackage[italian]{babel}
\usepackage{graphicx}

%****************enlarge layout
\textheight     243.5mm
\topmargin      -20.0mm
\textwidth      480pt
\hoffset        -80pt
%*****************theorems and such
\newcounter{esnu}
\newenvironment{esercizio}{\medskip \noindent {\bf Esercizio\addtocounter{esnu}{1} \arabic{esnu}}}{}
\pagestyle{empty}
\newcommand{\liff}{\mathrel{\leftrightarrow}}   % Logical IFF Symbol
\newcommand{\metaiff}{\Longleftrightarrow}      %iff in metatheory

\begin{document}

%\begin{tabular}{llclcr}
% \hspace{-35pt} &{\bf COGNOME:} & \hspace{100pt}        &{\bf NOME:}    & \hspace{100pt}        &{\bf MATRICOLA:}%\hspace{35pt} \\
%\hline
%\end{tabular}
\begin{center} {\bf Esame di Programmazione II, 28 luglio 2017}\end{center}
%\`

Si consideri un'implementazione di un negozio, al quale \`e possibile
aggiungere dei prodotti in vendita e fare degli ordini di acquisto. Per esempio:

{\scriptsize\begin{verbatim}
public class Main {
  public static void main(String[] args) throws MissingProductException {
    Product bike = new Product("bike", 300.0, 7); // una bicicletta costa 300 euro ed e' disponibile per la spedizione in 7 giorni
    Product phone = new Product("phone", 129.9, 1); // un telefono costa 129.9 euro ed e' disponibile per la spedizione in un giorno
    Shop amazing = new Shop(); amazing.add(bike, 3); amazing.add(phone, 4); // il negozio amazing ha disponibili 3 bici e 4 telefoni
    // creiamo tre ordini, due semplici e uno dividendo i prodotti per attesa di spedizione
    Order order1 = new SimpleOrder(amazing, bike, phone, phone); // una bici e due telefoni
    Order order2 = new SplitOrder(amazing, phone, bike, phone); // una bici e due telefoni
    Order order3 = new SimpleOrder(amazing, bike, phone); // una bici e un telefono
    // effettuiamo i tre ordini e stampiamo le spedizioni che ne risultano
    printShipping("FIRST ORDER:", order1.ship());
    printShipping("SECOND ORDER:", order2.ship());
    printShipping("THIRD ORDER:", order3.ship());
  }

  private static void printShipping(String title, Iterable<Shipping> shippings) {
    System.out.println(title + '\n');
    int counter = 1;
    for (Shipping shipping: shippings)
      System.out.println("shipping #" + counter++ + '\n' + shipping);
    System.out.println();
  }
}
\end{verbatim}}
  
\noindent
stamper\`a:
{\scriptsize\begin{verbatim}
FIRST ORDER:

shipping #1
bike: 300.00 euros, available in 7 days
phone: 129.90 euros, available in 1 days
phone: 129.90 euros, available in 1 days

SECOND ORDER:

shipping #1
phone: 129.90 euros, available in 1 days
phone: 129.90 euros, available in 1 days

shipping #2
bike: 300.00 euros, available in 7 days

Exception in thread "main" it.univr.ecommerce.MissingProductException
...
\end{verbatim}}

\noindent
terminando con un'eccezione. Si noti che un \texttt{SimpleOrder} viene spedito mettendo
tutti i prodotti in un'unica spedizione, mentre uno \texttt{SplitOrder} fa tante spedizioni,
dividendo i prodotti per giorni di attesa prima della loro disponibilit\`a alla spedizione.
La spedizione dell'ultimo \texttt{SimpleOrder} va in eccezione poich\'e non ci sono pi\`u
telefoni disponibile nel negozio, essendo stati gi\`a tutti spediti con i primi due ordini.

\vspace*{2ex}
\hrule

\begin{esercizio}
\textbf{[4 punti]}
Si completi la seguente classe \texttt{Product}:
%
{\small\begin{verbatim}
public class Product {
  private final String name;
  private final double price;
  private final int daysBeforeShipping;

  public Product(String name, double price, int daysBeforeShipping) { ... }

  @Override public String toString() {
    return String.format("%s: %.2f euros, available in %d days", name, price, daysBeforeShipping);
  }

  @Override public boolean equals(Object other) { ...confronta tutti e tre i campi }
  @Override public int hashCode() { ...consistente con equals() e non banale }

  public int getDaysBeforeShipping() {
    return daysBeforeShipping;
  }
}
\end{verbatim}}
%
\end{esercizio}

\begin{esercizio}
\textbf{[9 punti]}
Si completi la seguente classe che rappresenta un negozio a cui \`e possibile aggiungere prodotti
in vendita e da cui \`e possibile comprare prodotti (se disponibili):

{\small\begin{verbatim}
public class Shop {
  ...
  public void add(Product product, int howMany) {
    ...aggiunge howMany volte il prodotto indicato, che poteva gia' essere presente in negozio
  }

  void buy(Product[] productsToBuy) throws MissingProductException {
    ...rimuove i prodotti indicati da quelli disponibili in questo negozio;
    se non fossero disponibili tutti i prodotti, lancia una MissingProductException;
    in tal caso, il negozio dovra' restare immutato e nessun prodotto dovra' venire tolto
  }
}
\end{verbatim}}
%
\end{esercizio}

\begin{esercizio}
\textbf{[1 punti]}
Si implementi l'eccezione controllata \texttt{MissingProductException}.  
\end{esercizio}

\begin{esercizio}
\textbf{[3 punti]}
Si completi la seguente classe, che rappresenta una spedizione di alcuni prodotti:

{\small\begin{verbatim}
public class Shipping {
  ...
  Shipping(Iterable<Product> products) { ...crea una spedizione dei prodotti indicati }
  @Override public String toString() { ...ritorna la concatenzione del toString() dei prodotti spediti }
}
\end{verbatim}}
%
\end{esercizio}

\begin{esercizio}
\textbf{[5 punti]}
Si consideri la classe astratta che rappresenta un ordine da passare a un negozio per acquistare certi prodotti:
%
{\small\begin{verbatim}
public abstract class Order {
  private final Shop shop; // il negozio a cui si passa l'ordine
  private final Product[] products; // i prodotti che si vuole acquistare in tale negozio

  protected Order(Shop shop, Product... products) {
    this.shop = shop; this.products = products;
  }

  protected final Product[] getProducts() { return products; }

  // acquista i prodotti di questo ordine, andando in eccezione se non sono tutti disponibili in negozio
  protected final void buy() throws MissingProductException { shop.buy(products); }

  // acquista i prodotti di questo ordine, andando in eccezione se non sono tutti disponibili in negozio,
  // e ritorna le spedizioni da fare per inviarli
  public abstract Iterable<Shipping> ship() throws MissingProductException;
}
\end{verbatim}}

\noindent
Se ne completi la sottoclasse che spedisce tutti i prodotti in un'unica spedizione:
%
{\small\begin{verbatim}
public class SimpleOrder extends Order {
  public SimpleOrder(Shop shop, Product... products) { ... }
  @Override public Iterable<Shipping> ship() throws MissingProductException {
    ...compra i prodotti e ritorna un iterabile con un'unica spedizione
  }
}
\end{verbatim}}
%
\end{esercizio}

\begin{esercizio}
\textbf{[9 punti]}
Si completi l'altra sottoclasse di \texttt{Order}, che spedice i prodotti da acquistare in
tante spedizioni, dividendoli per giorni di attesa prima della loro disponibilit\`a all'invio:
%
{\small\begin{verbatim}
public class SplitOrder extends Order {
  public SplitOrder(Shop shop, Product... products) { ... }
  @Override public Iterable<Shipping> ship() throws MissingProductException {
    ...compra i prodotti e ritorna una o piu' spedizioni
  }
}
\end{verbatim}}
%
\end{esercizio}

\begin{center}
\textbf{\`E possibile definire campi, metodi, costruttori e classi aggiuntive, ma solo \texttt{private}.}
\end{center}

\end{document}
