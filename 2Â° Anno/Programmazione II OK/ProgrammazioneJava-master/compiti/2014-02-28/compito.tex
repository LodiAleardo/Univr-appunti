\documentclass{article}[10pt]
\usepackage[pdftex]{graphicx}
\usepackage{amsfonts}
\usepackage[italian]{babel}
%****************enlarge layout
\textheight     243.5mm
\topmargin      -20.0mm
\textwidth      480pt
\hoffset        -80pt
%*****************theorems and such
\newcounter{esnu}
\newenvironment{esercizio}{\medskip \noindent {\bf Esercizio\addtocounter{esnu}{1} \arabic{esnu}}}{}
\pagestyle{empty}
\newcommand{\liff}{\mathrel{\leftrightarrow}}   % Logical IFF Symbol
\newcommand{\metaiff}{\Longleftrightarrow}      %iff in metatheory

\begin{document}

%\begin{tabular}{llclcr}
% \hspace{-35pt} &{\bf COGNOME:} & \hspace{100pt}        &{\bf NOME:}    & \hspace{100pt}        &{\bf MATRICOLA:}%\hspace{35pt} \\
%\hline
%\end{tabular}
\begin{center} {\bf Esame di Programmazione II, 28 febbraio 2014}\end{center}
%\`

\begin{esercizio}
\textbf{[5 punti]}
%
Si completi la seguente classe, che implementa un momento temporale preciso fino al minuto:

{\small
\begin{verbatim}
public class Time implements Comparable<Time> {
  ...
  public Time(int year, int month, int day, int hours, int minutes) {
    ...
  }

  @Override
  public int compareTo(Time other) {
    ...
  }

  public String toStringOnlyHour() {
    ... ritorna stringhe del tipo hh:mm (solo ore e minuti)
  }
}
\end{verbatim}}

\noindent
Si noti che comparare due istanti temporali significa metterli in ordine temporale (chi viene
prima nel tempo).
\end{esercizio}

\begin{esercizio}
\textbf{[4 punti]}
%
Si completi la seguente classe, che implementa un treno che parte a un certo istante
temporale, ha un certo numero identificativo e una data destinazione:
%
{\small
\begin{verbatim}
public class Train implements Comparable<Train> {
  ...
  public Train(Time time, int number, String destination) { ... }

  @Override
  public String toString() {
    return String.format("%s : treno %d per %s", time.toStringOnlyHour(), number, destination);
  }

  @Override
  public int compareTo(Train other) {
    ... li mette in ordine temporale di partenza. A parita' di partenza, li mette
    in ordine crescente per numero di treno
  }
}
\end{verbatim}}
\end{esercizio}

\begin{esercizio}
\textbf{[3 punti]}
Si completi la seguente classe astratta, che rappresenta un cartellone che enumera (se iterato)
i treni in partenza:
%
{\small
\begin{verbatim}
public abstract class Cartellone implements Iterable<Train> {

  @Override
  public final String toString() {
    ... ritorna la concatenazione del toString() di tutti i treni nel cartellone
  }
}

\end{verbatim}}

\end{esercizio}

\begin{esercizio}
\textbf{[5 punti]}
Si completi la seguente classe, che implementa un cartellone al quale \`e possibile
aggiungere treni in partenza. I treni devono essere visualizzati in ordine crescente
secondo il loro metodo \texttt{compareTo()}:
%
{\small
\begin{verbatim}
public class CartelloneModificabile extends Cartellone {
  ...
  public CartelloneModificabile() {}
  public void add(Train train) { ...aggiunge train a questo cartellone }
  public void add(Iterable<Train> trains) { ...aggiunge tutti i trains a questo cartellone }
  public void add(Train... trains) { ...aggiunge tutti i trains a questo cartellone }

  @Override
  public Iterator<Train> iterator() { ...ritorna un iteratore sui trains di questo cartellone }
}
\end{verbatim}}

\noindent\textbf{Suggerimento:} la classe di libreria
\texttt{java.util.TreeSet} implementa un insieme che, se iterato, restituisce i suoi
elementi in ordine crescente rispetto al loro \texttt{compareTo}. Ha un metodo \texttt{add(elemento)}
che aggiunge l'elemento all'insieme.
\end{esercizio}

\begin{esercizio}
\textbf{[5 punti]}
Si completi la seguente classe, che implementa un cartellone che mostra solo i primi
\texttt{max} treni (al massimo) mostrati da un altro cartellone. In altri termini, iterando
su un \texttt{CartelloneLimitato} si ottengono i primi \texttt{max} treni (al massimo)
del cartellone originale.
%
{\small
\begin{verbatim}
public class CartelloneLimitato extends Cartellone {
  ...
  public CartelloneLimitato(Cartellone original, int max) { ... }

  @Override
  public Iterator<Train> iterator() {
    ...ritorna un iteratore sui primi max treni di original (al massimo)
  }
}
\end{verbatim}}
\end{esercizio}

\hrule

\vspace*{1ex}
Se tutto \`e corretto, l'esecuzione del seguente \texttt{main}:
%
{\small
\begin{verbatim}
public class Main {

  public static void main(String[] args) {
    CartelloneModificabile c = new CartelloneModificabile();

    // treno 1549 per Venezia delle 10:30 del 28/2/2014
    c.add(new Train(new Time(2014, 28, 2, 10, 30), 1549, "Venezia"));

    // treno 1804 per Bologna delle 11:45 del 28/2/2014
    c.add(new Train(new Time(2014, 28, 2, 11, 45), 1804, "Bologna"));

    // treno 1802 per Milano delle 11:45 del 28/2/2014
    c.add(new Train(new Time(2014, 28, 2, 11, 45), 1802, "Milano"));

    // treno 211 per Trieste delle 12:01 del 28/2/2014
    c.add(new Train(new Time(2014, 28, 2, 12, 01), 211, "Trieste"));

    // treno 1561 per Venezia delle 11:59 del 28/2/2014
    c.add(new Train(new Time(2014, 28, 2, 11, 59), 1561, "Venezia"));

    System.out.println(c);
    Cartellone lim = new CartelloneLimitato(c, 3);
    System.out.println("\nI prossimi 3 treni");
    System.out.println(lim);
  }
}
\end{verbatim}}

\noindent
deve stampare:

{\small
\begin{verbatim}
10:30 : treno 1549 per Venezia
11:45 : treno 1802 per Milano
11:45 : treno 1804 per Bologna
11:59 : treno 1561 per Venezia
12:01 : treno 211 per Trieste


I prossimi 3 treni
10:30 : treno 1549 per Venezia
11:45 : treno 1802 per Milano
11:45 : treno 1804 per Bologna
\end{verbatim}}

\end{document}
