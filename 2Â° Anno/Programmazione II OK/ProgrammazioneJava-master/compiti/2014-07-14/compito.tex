\documentclass{article}[10pt]
\usepackage[pdftex]{graphicx}
\usepackage{amsfonts}
%****************enlarge layout
\textheight     243.5mm
\topmargin      -20.0mm
\textwidth      480pt
\hoffset        -80pt
%*****************theorems and such
\newcounter{esnu}
\newenvironment{esercizio}{\medskip \noindent {\bf Esercizio\addtocounter{esnu}{1} \arabic{esnu}}}{}
\pagestyle{empty}
\newcommand{\liff}{\mathrel{\leftrightarrow}}   % Logical IFF Symbol
\newcommand{\metaiff}{\Longleftrightarrow}      %iff in metatheory

\begin{document}

%\begin{tabular}{llclcr}
% \hspace{-35pt} &{\bf COGNOME:} & \hspace{100pt}        &{\bf NOME:}    & \hspace{100pt}        &{\bf MATRICOLA:}%\hspace{35pt} \\
%\hline
%\end{tabular}
\begin{center} {\bf Esame di Programmazione II, 14 luglio 2014}\end{center}
%\`

Gli identificatori di un linguaggio di programmazione sono spesso scritti
in \emph{camel-style}, come ad esempio \texttt{camelsAreSweet}, oppure in
\emph{snake-style}, come ad esempio \texttt{snakes\_are\_slow\_food}. Inoltre vengono
spesso usati nella loro versione \emph{progressiva}, come ad esempio
\texttt{camelsAreSweet\_113} oppure \texttt{snakes\_are\_slow\_food\_113}.

Si consideri la superclasse che definisce un \emph{identificatore}, il cui unico metodo
permette di tradurre un identificare in stringa:

{\small\begin{verbatim}
public interface Identifier {
  @Override public String toString();
}
\end{verbatim}}

\begin{esercizio}
\textbf{[9 punti]}
%
Si completi la seguente classe astratta, che implementa un identificatore composto da pi\`u parole.
I costruttori controllano che le parole componenti non siano vuote e contengano solo
caratteri alfabetici minuscoli. Inoltre, controllano che ci sia almeno una parola. Altrimenti
lanciano un'eccezione:

{\small
\begin{verbatim}
public abstract class AbstractMultiWordsIdentifier implements Identifier {
  private final String[] words;

  // Fallisce con un'eccezione se non c'e' alcuna parola o se c'e' una parola vuota
  // o se una parola contiene un carattere non alfabetico minuscolo
  protected AbstractMultiWordsIdentifier(String... words)
    throws NoWordsProvidedException, WordIsNotAlphabeticalLowercaseException, EmptyWordException { ... }

  // Fallisce con un'eccezione nelle stesse condizioni viste sopra
  protected AbstractMultiWordsIdentifier(Iterable<String> words)
    throws NoWordsProvidedException, WordIsNotAlphabeticalLowercaseException, EmptyWordException { ... }

  // restituisce le parole fornite al momento della costruzione
  protected final String[] getWords() {
    return words;
  }
}
\end{verbatim}}

\noindent
Si noti che si tratta di una classe astratta in cui il metodo \texttt{toString()} non \`e
ancora implementato.
\end{esercizio}

\begin{esercizio}
\textbf{[3 punti]}
%
Si implementino le tre eccezioni usate nel punto precedente, tutte sottoclassi dell'eccezione
unchecked \texttt{java.lang.IllegalArgumentException}. Le clausole \texttt{throws} usate
nell'esercizio precedente sono necessarie o possono essere eliminate senza problemi di compilazione?
\end{esercizio}

\begin{esercizio}
\textbf{[3 punti]}
Si completi la seguente classe che implementa un identificatore camel-style, formato da pi\`u
parole congiunte:
%
{\small
\begin{verbatim}
public class CamelStyleIdentifier extends AbstractMultiWordsIdentifier {
  public CamelStyleIdentifier(String... words)
    throws NoWordsProvidedException, WordIsNotAlphabeticalLowercaseException, EmptyWordException { ... }

  public CamelStyleIdentifier(Iterable<String> words)
    throws NoWordsProvidedException, WordIsNotAlphabeticalLowercaseException, EmptyWordException { ... }

  @Override public String toString() { ... }
}
\end{verbatim}}

\end{esercizio}

\begin{esercizio}
\textbf{[3 punti]}
Si completi la seguente classe che implementa un identificatore snake-style, formato da pi\`u
parole congiunte:
%
{\small
\begin{verbatim}
public class SnakeStyleIdentifier extends AbstractMultiWordsIdentifier {
  public SnakeStyleIdentifier(String... words)
    throws NoWordsProvidedException, WordIsNotAlphabeticalLowercaseException, EmptyWordException { ... }

  public SnakeStyleIdentifier(Iterable<String> words)
    throws NoWordsProvidedException, WordIsNotAlphabeticalLowercaseException, EmptyWordException { ... }

  @Override public String toString() { ... }
}
\end{verbatim}}

\end{esercizio}

\begin{esercizio}
\textbf{[4 punti]}
Si completi la seguente classe, che implementa una versione progressiva di un altro identificatore
\texttt{base},
in cui cio\`e viene aggiunto un numero non-negativo \texttt{num}
in fondo alla rappresentazione stringa dell'identificatore,
separandolo con un underscore:
%
{\small
\begin{verbatim}
public class ProgressiveIdentifier implements Identifier {
  ...
  public ProgressiveIdentifier(Identifier base, int num) throws NegativeProgressiveNumberException { ... }

  @Override public String toString() { ... }
}
\end{verbatim}}

\noindent
L'identificatore \texttt{base}, in fondo a cui si aggiunge il numero, pu\`o essere di qualsiasi
tipo, anche di una classe di identificatore al momento non ancora definita.

Il costruttore deve lanciare una \texttt{NegativeProgressiveNumberException} se si tenta di
usare un numero progressivo \texttt{num} negativo. Si scriva tale eccezione, sottoclasse di
\texttt{java.lang.IllegalArgumentException}.
\end{esercizio}

\vspace*{2ex}
\hrule

\vspace*{2ex}

Se tutto \`e corretto, un'esecuzione del seguente \texttt{main}:
%
{\small
\begin{verbatim}
public class Main {
  public static void main(String[] args) {
    Iterable<String> words = readWords();

    System.out.println("In camel-style, le parole che hai inserito diventano");
    System.out.println(new CamelStyleIdentifier(words));
    System.out.println("In snake-style, le parole che hai inserito diventano");
    System.out.println(new SnakeStyleIdentifier(words));
    System.out.println("Le loro versioni progressive 816 sono");
    System.out.println(new ProgressiveIdentifier(new CamelStyleIdentifier(words), 816));
    System.out.println(new ProgressiveIdentifier(new SnakeStyleIdentifier(words), 816));
  }

  private static Iterable<String> readWords() {
    List<String> words = new ArrayList<String>();
    Scanner keyboard = new Scanner(System.in);

    while (true) {
      String word = keyboard.nextLine();
      if (word.equals("END"))
        break;
      words.add(word);
    }

    keyboard.close();

    return words;
  }
}
\end{verbatim}}

\noindent
potrebbe essere:

{\small
\begin{verbatim}
camels    <-- input dell'utente
are       <-- input dell'utente
sweet     <-- input dell'utente
and       <-- input dell'utente
clever    <-- input dell'utente
animals   <-- input dell'utente
END       <-- input dell'utente
In camel-style, le parole che hai inserito diventano
CamelsAreSweetAndCleverAnimals
In snake-style, le parole che hai inserito diventano
camels_are_sweet_and_clever_animals
Le loro versioni progressive 816 sono
CamelsAreSweetAndCleverAnimals_816
camels_are_sweet_and_clever_animals_816
\end{verbatim}}

\end{document}
