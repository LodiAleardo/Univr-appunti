\documentclass{article}[10pt]
\usepackage[pdftex]{graphicx}
\usepackage{amsfonts}
\usepackage[italian]{babel}
%****************enlarge layout
\textheight     243.5mm
\topmargin      -20.0mm
\textwidth      480pt
\hoffset        -80pt
%*****************theorems and such
\newcounter{esnu}
\newenvironment{esercizio}{\medskip \noindent {\bf Esercizio\addtocounter{esnu}{1} \arabic{esnu}}}{}
\pagestyle{empty}
\newcommand{\liff}{\mathrel{\leftrightarrow}}   % Logical IFF Symbol
\newcommand{\metaiff}{\Longleftrightarrow}      %iff in metatheory

\begin{document}

%\begin{tabular}{llclcr}
% \hspace{-35pt} &{\bf COGNOME:} & \hspace{100pt}        &{\bf NOME:}    & \hspace{100pt}        &{\bf MATRICOLA:}%\hspace{35pt} \\
%\hline
%\end{tabular}
\begin{center} {\bf Esame di Programmazione II, 23 febbraio 2015}\end{center}
%\`

Il gioco della vita \`e una tabella bidimensionale di cellule, le quali possono essere vive
(rappresentante con un asterisco) oppure morte (rappresentate con uno spazio).
Per esempio, la seguente \`e una  rappresentazione di un gioco della vita di dimensione
$40\times 7$, in un certo istante:

{\small
\begin{verbatim}
                                        
  **                                    
  **                      **            
                          * *           
                          *             
                                        
     ***                                
\end{verbatim}}

\`E possibile cambiare la tabella di cellule secondo due regole:
%
\begin{enumerate}
\item una cellula morta con esattamente 3 vicini vivi viene sostituita con una cellula viva;
\item una cellula viva con 2 o 3 vicini vivi viene sostituita con una cellula viva;
      altrimenti viene sostituita con una cellula morta (per isolamento o sovraffollamento).
\end{enumerate}

\begin{esercizio}
\textbf{[6 punti]}
%
Si completi la seguente classe, che implementa una cellula.
%
{\small
\begin{verbatim}
public class Cell {
  private final boolean alive; // viva o morta?
  public Cell(boolean alive) { ... } // costruisce una cellula, specificando se viva o morta
  public String toString() { ... } // stringa di un solo carattere, asterisco o spazio
  public Cell next(Iterable<Cell> neighbors) { ... }
}
\end{verbatim}}

\noindent
Il metodo \texttt{next} restituisce la cellula che deve sostituire \texttt{this}, implementando le due
regole del gioco date sopra. Il parametro \texttt{neighbors} sono i vicini di \texttt{this},
la cui vita o morte determina la scelta della cellula che deve sostituire \texttt{this}.
\end{esercizio}

\begin{esercizio}
\textbf{[5 punti]}
%
Nel gioco della vita esistono delle \emph{figure}, cio\`e
delle combinazioni di cellule che prendono nomi ben precisi. Per
esempio, un \emph{blocco} \`e una figura fatta da quattro cellule vive:

{\small\begin{verbatim}
**
**
\end{verbatim}}

\noindent
La seguente classe implementa una figura, che conosce le coordinate di partenza, in alto a sinistra,
in cui la figura \`e posizionata dentro una tabella di cellule (le coordinate
crescono verso il basso e verso destra):

{\small\begin{verbatim}
public abstract class Figure {
  protected final int startX;
  protected final int startY;
  
  protected Figure(int startX, int startY) {
    this.startX = startX;
    this.startY = startY;
  }

  public abstract Cell mkAliveCellAt(int x, int y);
}
\end{verbatim}}

\noindent
Il metodo \texttt{mkAliveCellAt} restituisce la cellula alle coordinate $x$ e $y$ della
tabella di cellule solo se appartiene alla figura ed \`e viva; altrimenti ritorna \texttt{null}.
Per esempio, la seguente \`e l'implementazione di un blocco:
%
\newpage
{\small\begin{verbatim}
public class Block extends Figure {
  public Block(int startX, int startY) {
    super(startX, startY);
  }

  @Override
  public Cell mkAliveCellAt(int x, int y) {
    x = x - startX; // calcola lo spostamento rispetto all'angolo in alto a sinistra della figura
    y = y - startY;

    // le cellule vive sono in coordinate tra 0 e 1, sia ascissa che ordinata
    return 0 <= x && x <= 1 && 0 <= y && y <= 1 ? new Cell(true) : null;
  }
}
\end{verbatim}}

Si scrivano altre sottoclassi di \texttt{Figura}, in modo da implementare:
\begin{itemize}
\item un \emph{lampeggiatore} di dimensione $3\times 1$ (classe \texttt{Blinker.java}):
{\small\begin{verbatim}
***
\end{verbatim}}      
\item una \emph{barca} di dimensione $3\times 3$ (classe \texttt{Ship.java}):
{\small\begin{verbatim}
**
* *
 *
\end{verbatim}}
\item un \emph{rospo} di dimensione $4\times 2$ (classe \texttt{Toad.java}):
{\small\begin{verbatim}
 ***
***
\end{verbatim}}
\item un \emph{aliante} di dimensione $3\times 3$ (classe \texttt{Glider.java}):
{\small\begin{verbatim}
***
*
 *
\end{verbatim}}      
\end{itemize}

\end{esercizio}

\begin{esercizio}
\textbf{[11 punti]}
La tabella di cellule contiene le cellule del gioco e permette di sostituirle con quelle
che risultano dalla due regole viste sopra. Si completi la seguente classe che
implementa la tabella:

{\small
\begin{verbatim}
public class Board {
  private final int width;
  private final int height;
  private final Cell[][] cells;

  // costruisce una tabella delle dimensioni indicate e in cui sono posizionate
  // le figure indicate. Altrove le cellule devono essere inizialmente morte
  public Board(int width, int height, Figure... figures) { ... }

  // fa passare le cellule della tabella alla loro configurazione successiva
  public void recomputeCells() { ... }

  // restituisce una descrizione della tabella, fatta da spazi e asterischi,
  // come quella mostrata all'inizio del compito
  public String toString() { ... }
}
\end{verbatim}}

\end{esercizio}

\end{document}
