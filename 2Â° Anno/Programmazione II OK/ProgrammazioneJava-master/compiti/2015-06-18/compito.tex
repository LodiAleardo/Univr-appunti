\documentclass{article}[10pt]
\usepackage[pdftex]{graphicx}
\usepackage{amsfonts}
\usepackage[italian]{babel}
%****************enlarge layout
\textheight     243.5mm
\topmargin      -20.0mm
\textwidth      480pt
\hoffset        -80pt
%*****************theorems and such
\newcounter{esnu}
\newenvironment{esercizio}{\medskip \noindent {\bf Esercizio\addtocounter{esnu}{1} \arabic{esnu}}}{}
\pagestyle{empty}
\newcommand{\liff}{\mathrel{\leftrightarrow}}   % Logical IFF Symbol
\newcommand{\metaiff}{\Longleftrightarrow}      %iff in metatheory

\begin{document}

%\begin{tabular}{llclcr}
% \hspace{-35pt} &{\bf COGNOME:} & \hspace{100pt}        &{\bf NOME:}    & \hspace{100pt}        &{\bf MATRICOLA:}%\hspace{35pt} \\
%\hline
%\end{tabular}
\begin{center} {\bf Esame di Programmazione II, 18 giugno 2015}\end{center}
%\`

La libreria Java standard contiene un'interfaccia che specifica una sequenza di caratteri di cui \`e possibile
estrarre sottosequenze:

{\small
\begin{verbatim}
public interface java.lang.CharSequence {
  int length();
  char charAt(int pos);
  java.lang.CharSequence subSequence(int start, int end);
  java.lang.String toString();
}
\end{verbatim}
}

\noindent
Il metodo \texttt{length()} restituisce il numero di caratteri della sequenza; il metodo \texttt{charAt()}
restituisce il \texttt{pos}-esimo carattere della sequenza e genera una \texttt{IndexOutOfBoundsException} se
\texttt{pos} \`e negativo o maggiore o uguale di \texttt{length()}; il metodo \texttt{subSequence()}
ritorna una sottosequenza dal carattere \texttt{start}-esimo (incluso) al carattere \texttt{end}-esimo (escluso)
e genera una \texttt{IndexOutOfBoundsException} se \texttt{start} o \texttt{end} sono
negativi o se \texttt{end} \`e maggiore di \texttt{length()} o se \texttt{start} \`e maggiore di \texttt{end};
il metodo \texttt{toString()} ritorna una stringa formata dalla concatenazione dei caratteri della sequenza.
Si noti che \texttt{java.lang.String} implementa \texttt{java.lang.CharSequence}.

\begin{esercizio}
\textbf{[4 punti]}
%
Si implementi una classe astratta \texttt{AbstractCharSequence}
che implementa \texttt{CharSequence} lasciandone tutti i metodi
astratti, tranne \texttt{toString()} che viene implementato e specificato come \texttt{final}.
\end{esercizio}

\begin{esercizio}
\textbf{[9 punti]}
%
Si completi la seguente sottoclasse di \texttt{AbstractCharSequence},
che specifica una sequenza di caratteri alfabetici che parte da un
dato carattere \texttt{start} e continua con i successivi caratteri
minuscoli dell'alfabeto inglese, ciclicamente, per una certa lunghezza
specificata al costruttore: dopo la \texttt{'z'} si ricomincia dalla \texttt{'a'}.
Nel costruttore, \texttt{start} deve essere tra \texttt{'a'} e
\texttt{'z'} altrimenti deve venire lanciata una \texttt{IllegalArgumentException};
il parametro \texttt{length} deve essere non negativo, altrimenti deve
venire lanciata una \texttt{IndexOutOfBoundsException}:

{\small\begin{verbatim}
public class Alphabetical extends AbstractCharSequence {
  ...
  public Alphabetical(char start, int length) { ... }

  @Override
  public int length() { ... }

  @Override
  public char charAt(int index) { ... }

  @Override
  public Alphabetical subSequence(int start, int end) { ... }
}
\end{verbatim}}

\noindent
Si noti che il tipo di ritorno del metodo \texttt{subSequence()} \`e stato raffinato e questo
dovr\`a venire rispettato dalla sua implementazione.
\end{esercizio}

\begin{esercizio}
\textbf{[9 punti]}
Si completi la seguente sottoclasse di \texttt{AbstractCharSequence},
che permette di costruire una sequenza identica a quella fonita al suo
costruttore,
ma con in pi\`u un carattere di controllo alfabetico in fondo. Tale carattere
\`e definito come la somma dei codici dei caratteri della sequenza
originale, modulo $26$, pi\`u il codice di \texttt{'a'}.

{\small
\begin{verbatim}
public class ControlCode extends AbstractCharSequence {
  ...
  public ControlCode(CharSequence original) { ... }

  @Override
  public int length() { ... }

  @Override
  public char charAt(int index) { ... }

  @Override
  public CharSequence subSequence(int start, int end) { ... }
}
\end{verbatim}}

\end{esercizio}

\newpage

Se tutto \`e corretto, l'esecuzione del seguente programma:

{\small\begin{verbatim}
public class Main {

  private static int counter;

  public static void main(String[] args) {
    CharSequence a1 = new Alphabetical('c', 26);
    CharSequence a2 = new Alphabetical('h', 60);
    CharSequence a3 = a2.subSequence(5, 40);
    print(a1);  // 1
    print(a2);  // 2
    print(a3);  // 3

    CharSequence c1 = new ControlCode("Shakespeare");
    CharSequence c2 = new ControlCode(a3);
    print(c1);  // 4
    print(c2);  // 5

    CharSequence s1 = "There is no world without Verona walls";
    CharSequence s2 = "But purgatory, torture, hell itself";
    print(s1);  // 6
    print(s2);  // 7

    CharSequence app = s1.toString() + a2.toString();
    print(app); // 8

    CharSequence last = c2.subSequence(35, 36);
    print(last);// 9
  }

  private static void print(CharSequence seq) {
    System.out.println(++counter + ": " + seq + " length = " + seq.length());
  }
}
\end{verbatim}}

\noindent
dovr\`a stampare:

{\small\begin{verbatim}
1: cdefghijklmnopqrstuvwxyzab length = 26
2: hijklmnopqrstuvwxyzabcdefghijklmnopqrstuvwxyzabcdefghijklmno length = 60
3: mnopqrstuvwxyzabcdefghijklmnopqrstu length = 35
4: Shakespeareo length = 12
5: mnopqrstuvwxyzabcdefghijklmnopqrstuq length = 36
6: There is no world without Verona walls length = 38
7: But purgatory, torture, hell itself length = 35
8: There is no world without Verona wallshijklmnopqrstuvwxyzabcdefghijklmnopqrstuvwxyzabcdefghijklmno length = 98
9: q length = 1
\end{verbatim}}

\end{document}
