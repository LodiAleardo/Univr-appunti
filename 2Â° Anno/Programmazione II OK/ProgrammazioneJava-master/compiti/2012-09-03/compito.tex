\documentclass{article}[10pt]
\usepackage[pdftex]{graphicx}
\usepackage{amsfonts}
\usepackage[italian]{babel}
%****************enlarge layout
\textheight     245.0mm
\topmargin      -22.0mm
\textwidth      480pt
\hoffset        -80pt
%*****************theorems and such
\newcounter{esnu}
\newenvironment{esercizio}{\medskip \noindent {\bf Esercizio\addtocounter{esnu}{1} \arabic{esnu}}}{}
\pagestyle{empty}
\newcommand{\liff}{\mathrel{\leftrightarrow}}   % Logical IFF Symbol
\newcommand{\metaiff}{\Longleftrightarrow}      %iff in metatheory

\begin{document}

%\begin{tabular}{llclcr}
% \hspace{-35pt} &{\bf COGNOME:} & \hspace{100pt}        &{\bf NOME:}    & \hspace{100pt}        &{\bf MATRICOLA:}%\hspace{35pt} \\
%\hline
%\end{tabular}
\begin{center} {\bf Esame di Programmazione II, 3 settembre 2012}\end{center}

Si considerino le seguenti interfacce:

{\small
\begin{verbatim}
public interface Set extends Iterable<Object> {
  boolean contains(Object element);
  boolean intersects(Set other);
  int size();
}

public interface ModifiableSet extends Set {
  boolean add(Object element);
  boolean remove(Object element);
  boolean addAll(Set set);
  boolean removeAll(Set set);
}
\end{verbatim}
}

\noindent
La prima specifica l'interfaccia di un insieme senza metodi di modifica, mentre la seconda
quella di un insieme con metodi di modifica.

\begin{esercizio}
\textbf{[11 punti]}

Si scriva una classe \texttt{ArraySet} che implementa \texttt{Set}, utilizza un array per contenere
i suoi elementi e implementa i seguenti costruttori e metodi:

{\small
\begin{verbatim}
public ArraySet(Object... elements)
protected Object[] getElements()
protected void setElements(Object[] elements)
public boolean contains(Object element)
public boolean intersects(Set other)
public int size()
public Iterator<Object> iterator()
public boolean equals(Object other)
public int hashCode()
\end{verbatim}
}

\noindent
Si noti che il costruttore costruisce un insieme non modificabile che contiene gli elementi forniti.
I metodi \texttt{getElements} e \texttt{setElements} forniscono accesso in lettura e scrittura
agli elementi contenuti nell'insieme.
Il metodo \texttt{contains} determina se un elemento \`e contenuto nell'insieme.
Il metodo \texttt{intersects} determina se \texttt{this} interseca \texttt{other}.
Il metodo \texttt{size} restituisce il numero di elementi contenuti nell'insieme.
Il metodo \texttt{iterator} restituisce un iteratore sugli elementi dell'insieme. Il metodo
\texttt{equals} determina se \texttt{this} e \texttt{other} sono due \texttt{Set} che contengono gli stessi
elementi, in qualsiasi ordine. Il metodo \texttt{hashCode} deve essere non banale e consistente con \texttt{equals}.

L'uguaglianza fra gli elementi degli insiemi deve essere determinata dal metodo \texttt{equals} di tali elementi,
non da \texttt{==}.
\end{esercizio}

\begin{esercizio}
\textbf{[11 punti]}
Si scriva una classe \texttt{ModifiableArraySet} che implementa \texttt{ModifiableSet} ed estende
\texttt{ArraySet}. Oltre ai metodi di \texttt{ArraySet}, deve implementare quindi anche i metodi
e costruttori:

{\small
\begin{verbatim}
public ModifiableArraySet()
public ModifiableArraySet(Object... elements)
public ModifiableArraySet(Set father)
public boolean add(Object element)
public boolean remove(Object element)
public boolean addAll(Set set)
public boolean removeAll(Set set)
\end{verbatim}}

\noindent
Si noti che il costruttore senza argomenti costruisce un insieme inizialmente vuoto.
Il costruttore che riceve come argomento
un altro insieme \texttt{father} costruisce un insieme modificabile
che contiene inizialmente gli stessi elementi di \texttt{father}.
Le funzioni che aggiungono o rimuovono uno o pi\`u elementi restituiscono \texttt{true} se e solo se
viene effettivamente aggiunto o rimosso un elemento. Per esempio, se si aggiunge un elemento a un insieme
che gi\`a lo contiene, \texttt{add} deve restituire \texttt{false}; se si rimuove un elemento da un
insieme che non lo contiene, \texttt{remove} deve restituire \texttt{false}.
\end{esercizio}

\pagebreak
Se tutto \`e corretto, l'esecuzione del seguente programma:

{\small
\begin{verbatim}
public class Main {

  public static void main(String[] args) {
    Set s1 = new ArraySet("ciao", "amico", "come", "va?");
    Set s2 = new ModifiableArraySet("oggi", "va?", 12, 113);

    System.out.println("1: " + s1.equals(s2));
    System.out.println("2: " + s1.intersects(s2));

    ModifiableSet s3 = new ModifiableArraySet("amico", "va?", "ciao", "va?");
    s3.add("come");
    s3.add(new String("ciao"));

    System.out.println("3: " + s1.equals(s3));
    System.out.println("4: " + s1.intersects(s3));
  }
}
\end{verbatim}
}

\noindent
dovr\`a stampare:

{\small
\begin{verbatim}
1: false
2: true
3: true
4: true
\end{verbatim}
}

\end{document}
