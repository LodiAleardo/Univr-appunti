\documentclass{article}[10pt]
\usepackage[pdftex]{graphicx}
\usepackage{amsfonts}
\usepackage[italian]{babel}
%****************enlarge layout
\textheight     243.5mm
\topmargin      -20.0mm
\textwidth      480pt
\hoffset        -80pt
%*****************theorems and such
\newcounter{esnu}
\newenvironment{esercizio}{\medskip \noindent {\bf Esercizio\addtocounter{esnu}{1} \arabic{esnu}}}{}
\pagestyle{empty}
\newcommand{\liff}{\mathrel{\leftrightarrow}}   % Logical IFF Symbol
\newcommand{\metaiff}{\Longleftrightarrow}      %iff in metatheory

\begin{document}

%\begin{tabular}{llclcr}
% \hspace{-35pt} &{\bf COGNOME:} & \hspace{100pt}        &{\bf NOME:}    & \hspace{100pt}        &{\bf MATRICOLA:}%\hspace{35pt} \\
%\hline
%\end{tabular}
\begin{center} {\bf Esame di Programmazione II, 26 febbraio 2013 (2 ore)}\end{center}

\begin{esercizio}
\textbf{[6 punti]}
Si completi la seguente classe \texttt{Time} che rappresenta un istante temporale.
Due istanti temporali devono essere \texttt{equals} se rappresentano lo stesso momento,
esattamente. Il metodo \texttt{hashCode} deve essere non banale. Il metodo
\texttt{compareTo} deve esprimere l'ordinamento cronologico fra istanti temporali.
Il metodo \texttt{toString} deve restituire una stringa del tipo
\textit{yy/mm/dd, hh:mm}, ad esempio \texttt{14/2/2013, 21:00}.
%
{\small
\begin{verbatim}
public class Time implements Comparable<Time> {

  public Time(int year, int month, int day, int hour, int minute) { .... }

  @Override
  public String toString() { .... }

  @Override
  public boolean equals(Object other) { .... }

  @Override
  public int hashCode() { .... }

  @Override
  public int compareTo(Time other) { .... }
}
\end{verbatim}
}
\end{esercizio}

\begin{esercizio}
\textbf{[6 punti]}
Si completi la seguente classe, che implementa un evento che comincia
a un certo istante temporale e ha un nome che lo descrive:
%
{\small
\begin{verbatim}
public abstract class Event implements Comparable<Event> {

  protected Event(Time startingTime, String name) { .... }

  public abstract void delete();

  @Override
  public final String toString() { .... }

  @Override
  public final int compareTo(Event other) { .... }

  @Override
  public final boolean equals(Object other) { .... }

  @Override
  public final int hashCode() { .... }
}
\end{verbatim}}

\noindent
Il metodo \texttt{toString} restituisce una stringa che fornisce il
nome dell'evento e il momento di inizio, del tipo
\texttt{uscire con la fidanzata @ 14/2/2013, 21:00}.
Due eventi sono \texttt{equals} se hanno stesso nome e stesso
momento di inizio. Il metodo \texttt{hashCode} deve
essere non banale. Il metodo \texttt{compareTo} deve ordinare
gli eventi per momento di inizio; a parit\`a di momento
di inizio, li deve ordinare in ordine alfabetico per nome.
\textbf{Suggerimento: } \texttt{String} implementa
\texttt{java.lang.Comparable<String>}.

Il metodo \texttt{delete} \`e astratto e dovr\`a quindi essere
implementato nelle sottoclassi concrete di \texttt{Event}.
Il suo effetto sar\`a di cancellare l'evento, eliminandolo
dal calendario a cui appartiene.
\end{esercizio}

\begin{esercizio}
\textbf{[10 punti]}
Si definisca una classe \texttt{Calendar} che implementa una collezione di eventi
temporali (un calendario quindi). Deve fornire:
%
\begin{itemize}
\item un costruttore vuoto, che costruisce un calendario inizialmente privo di eventi;
\item un metodo \texttt{addEvent(Time startingTime, String name)} che prende nota
      che al momento indicato inizia un evento dal nome indicato;
\item un metodo \texttt{toString} che restituisce una stringa che enumera gli
      eventi in calendario, su righe successive.
\end{itemize}
%
La classe \texttt{Calendar} deve essere iterabile (\texttt{java.lang.Iterable}), nel senso che
iterando su un calendario si ottengono gli eventi (classe \texttt{Event})
in calendario, in ordine
cronologico (dal pi\`u vecchio al pi\`u nuovo). Quindi avr\`a un altro
metodo pubblico oltre a quelli enumerati sopra.

\textbf{Suggerimento: } la classe iterabile di libreria
\texttt{java.util.TreeSet<C>} realizza un insieme
ordinato purch\'e \texttt{C} implementi \texttt{java.lang.Comparable<C>}.
L'ordinamento avviene rispetto al metodo \texttt{compareTo} degli elementi
dell'insieme. Gli elementi vengono aggiunti con il metodo
\texttt{add(C elemento)} (se l'elemento c'\`e gi\`a, non fa nulla)
e rimossi con il metodo
\texttt{remove(C elemento)} (se l'elemento non c'\`e, non fa nulla).
\end{esercizio}

\vspace*{3ex}
\hrule
\vspace*{3ex}
Se tutto \`e corretto, il seguente programma:
%
{\small
\begin{verbatim}
import java.util.HashSet;
import java.util.Set;

public class Main {

  public static void main(String[] args) {
    Calendar cal = new Calendar();
    cal.addEvent(new Time(2013, 2, 14, 21, 00), "uscire con la fidanzata");
    cal.addEvent(new Time(2013, 2, 26, 12, 45), "andare in mensa");
    cal.addEvent(new Time(2013, 2, 27, 9, 10), "andare a sciare");
    cal.addEvent(new Time(2013, 2, 26, 10, 00), "esame di Programmazione II");
    cal.addEvent(new Time(2013, 2, 26, 12, 45), "chiamare la mamma");
    cal.addEvent(new Time(2013, 3, 4, 8, 30), "ricominciare ad andare a lezione");

    System.out.println(cal);

    // ok, la fidanzata e' andata a sciare con qualcun altro:
    // collezioniamo tutti gli eventi che riguardano sciare o che hanno
    // a che fare con la ex-fidanzata....
    Set<Event> deleteThese = new HashSet<Event>();
    for (Event e: cal)
      if (e.toString().contains("sciare") || e.toString().contains("fidanzata"))
        deleteThese.add(e);

    // ....e rimuoviamoli!
    for (Event e: deleteThese)
      e.delete();

    System.out.println(cal);
  }
}
\end{verbatim}}

\noindent
stamper\`a:
%
{\small
\begin{verbatim}
uscire con la fidanzata @ 14/2/2013, 21:00
esame di Programmazione II @ 26/2/2013, 10:00
andare in mensa @ 26/2/2013, 12:45
chiamare la mamma @ 26/2/2013, 12:45
andare a sciare @ 27/2/2013, 9:10
ricominciare ad andare a lezione @ 4/3/2013, 8:30

esame di Programmazione II @ 26/2/2013, 10:00
andare in mensa @ 26/2/2013, 12:45
chiamare la mamma @ 26/2/2013, 12:45
ricominciare ad andare a lezione @ 4/3/2013, 8:30
\end{verbatim}}
%
\end{document}
