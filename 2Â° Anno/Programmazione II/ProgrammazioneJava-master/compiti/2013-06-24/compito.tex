\documentclass{article}[10pt]
\usepackage[pdftex]{graphicx}
\usepackage{amsfonts}
\usepackage[italian]{babel}
%****************enlarge layout
\textheight     243.5mm
\topmargin      -20.0mm
\textwidth      480pt
\hoffset        -80pt
%*****************theorems and such
\newcounter{esnu}
\newenvironment{esercizio}{\medskip \noindent {\bf Esercizio\addtocounter{esnu}{1} \arabic{esnu}}}{}
\pagestyle{empty}
\newcommand{\liff}{\mathrel{\leftrightarrow}}   % Logical IFF Symbol
\newcommand{\metaiff}{\Longleftrightarrow}      %iff in metatheory

\begin{document}

%\begin{tabular}{llclcr}
% \hspace{-35pt} &{\bf COGNOME:} & \hspace{100pt}        &{\bf NOME:}    & \hspace{100pt}        &{\bf MATRICOLA:}%\hspace{35pt} \\
%\hline
%\end{tabular}
\begin{center} {\bf Esame di Programmazione II, 24 giugno 2013 (2 ore)}\end{center}

Le note musicali sono normalmente divise in 12 \emph{semitoni}, come nella seguente tabella:

\begin{center}
\begin{tabular}{c|c|c}
  semitono & nome italiano & nome inglese \\\hline
  0 & do & C \\
  1 & do\# & C\#\\
  2 & re & D\\
  3 & re\# & D\#\\
  4 & mi & E \\
  5 & fa & F \\
  6 & fa\# & F\# \\
  7 & sol & G \\
  8 & sol\# & G\# \\
  9 & la & A \\
  10 & la\# & A\# \\
  11 & si & B
\end{tabular}
\end{center}

\begin{esercizio}
\textbf{[2 punti]}
Si completi la seguente classe che rappresenta una nota musicale:
%
{\small
\begin{verbatim}
public abstract class Nota {
  protected Nota(int semitono) { ... }

  @Override
  public abstract String toString();

  public abstract Nota incrementa(int inc);

  protected final int getSemitono() { ... }
}
\end{verbatim}
}

\noindent
dove \texttt{getSemitono()} restituisce il semitono della nota, che era stato passato al costruttore.
Il metodo \texttt{incrementa()}, che deve essere implemetato nelle sottoclassi concrete,
restituisce una nota ottenuta aumentando il semitono della nota di \texttt{inc} semitoni. Si noti
che la scala \`e circolare per cui, dopo il \emph{si}, si ritorna sul \emph{do}.
\end{esercizio}

\begin{esercizio}
\textbf{[4 punti]}
Si scrivano le sottoclassi concrete \texttt{NotaIT} e \texttt{NotaUK} di \texttt{Nota} che si differenziano
per il metodo \texttt{toString()}, che restituisce una rappresentazione della nota usando, rispettivamente,
la notazione italiana e quella inglese.
\end{esercizio}

\begin{esercizio}
\textbf{[4 punti]}
Si consideri l'interfaccia:
%
{\small
\begin{verbatim}
public interface Canzone extends Iterable<Nota> {
  public String getNome();
}
\end{verbatim}
}

\noindent
che implementa una canzone. Il metodo \texttt{getNome()} restituisce il nome delle canzone.
Iterando su una canzone si deve iterare sulle sue note.

Si implementi quindi una sottoclasse astratta di \texttt{Canzone}:
%
{\small
\begin{verbatim}
public abstract class AbstractCanzone implements Canzone {

  @Override
  public final String toString() { ... }
}
\end{verbatim}
}

\noindent
completando il metodo \texttt{toString()} in modo che restituisca il nome della canzone
e il nome delle note usate nella canzone, nel loro ordine di iterazione.
\end{esercizio}

\begin{esercizio}
\textbf{[5 punti]}
Si completi la seguente classe concreta:
%
{\small
\begin{verbatim}
public class CanzoneImpl extends AbstractCanzone {
  public CanzoneImpl(String nome, Nota... note) { ... }

  @Override
  public String getNome() { ... }

  @Override
  public Iterator<Nota> iterator() { ... }
}
\end{verbatim}
}

\noindent
in modo che al costruttore vengano passati il nome della canzone e la sequenza di note che
la compongono. Se si itera su questa canzone, saranno quelle le note che verranno ottenute dall'iterazione.
\end{esercizio}

\begin{esercizio}
\textbf{[7 punti]}
Si completi la seguente classe concreta:
%
{\small
\begin{verbatim}
public class CanzoneRibasata extends AbstractCanzone {
  public CanzoneRibasata(Canzone base, int inc) { ... }

  @Override
  public Iterator<Nota> iterator() { ... }

  @Override
  public String getNome() { ... }
}
\end{verbatim}
}

\noindent
che definisce una canzone identica a \texttt{base} ma le cui note sono spostate di \texttt{inc} semitoni
rispetto a quelle di \texttt{base}.
\end{esercizio}

\vspace*{3ex}
\hrule
\vspace*{3ex}
Se tutto \`e corretto, il seguente programma:
%
{\small
\begin{verbatim}
public class Main {
  public static void main(String[] args) {
    Canzone boccaDiRosa = new CanzoneImpl
      ("Bocca di Rosa", new NotaIT(11), new NotaIT(2), new NotaIT(3), new NotaIT(6));
    Canzone letItBe = new CanzoneImpl("Let it be", new NotaUK(3), new NotaUK(5), new NotaUK(0));

    System.out.println(boccaDiRosa);
    System.out.println(letItBe);
    System.out.println(new CanzoneRibasata(boccaDiRosa, 5));
    System.out.println(new CanzoneRibasata(letItBe, 23));
  }
}
\end{verbatim}}

\noindent
stamper\`a:
%
{\small
\begin{verbatim}
Bocca di Rosa:
si
re
re#
fa#

Let it be:
D#
F
C

Bocca di Rosa:
mi
sol
sol#
si

Let it be:
D
E
B
\end{verbatim}}
%
\end{document}
