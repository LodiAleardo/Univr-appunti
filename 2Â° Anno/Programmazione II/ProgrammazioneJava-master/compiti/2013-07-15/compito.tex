\documentclass{article}[10pt]
\usepackage[pdftex]{graphicx}
\usepackage{amsfonts}
\usepackage[italian]{babel}
%\`
%****************enlarge layout
\textheight     243.5mm
\topmargin      -20.0mm
\textwidth      480pt
\hoffset        -80pt
%*****************theorems and such
\newcounter{esnu}
\newenvironment{esercizio}{\medskip \noindent {\bf Esercizio\addtocounter{esnu}{1} \arabic{esnu}}}{}
\pagestyle{empty}
\newcommand{\liff}{\mathrel{\leftrightarrow}}   % Logical IFF Symbol
\newcommand{\metaiff}{\Longleftrightarrow}      %iff in metatheory

\begin{document}

%\begin{tabular}{llclcr}
% \hspace{-35pt} &{\bf COGNOME:} & \hspace{100pt}        &{\bf NOME:}    & \hspace{100pt}        &{\bf MATRICOLA:}%\hspace{35pt} \\
%\hline
%\end{tabular}
\begin{center} {\bf Esame di Programmazione II, 15 luglio 2013 (2 ore)}\end{center}

Un foglio elettronico \`e una tabella bidimensionale di celle. Dentro le celle \`e posizionata
un'espressione, il cui valore \`e visualizzato dentro la cella. L'espressione di
una cella pu\`o essere una costante numerica intera, una stringa, un'operazione aritmetica fra
due altre espressioni, la concatenazione fra due altre espressioni o un riferimento a una cella.
Per esempio, il seguente programma costruisce un foglio eletronico, popola alcune sue celle con
espressioni e infine lo stampa:

{\small
\begin{verbatim}
public class Main {
  public static void main(String[] args) {
    Sheet sheet = new Sheet(5, 6);
    sheet.setCell(1, 0, new NumericConstant(13));
    sheet.setCell(2, 0, new NumericConstant(-45));
    sheet.setCell(1, 1, new NumericConstant(11));
    sheet.setCell(2, 1, new NumericConstant(23));
    sheet.setCell(1, 2, new NumericConstant(1));
    sheet.setCell(2, 2, new NumericConstant(4));
    sheet.setCell(0, 3, new StringConstant("totale:"));
    sheet.setCell(1, 3, new Add(sheet.getCell(1, 0), new Add(sheet.getCell(1, 1), sheet.getCell(1, 2))));
    sheet.setCell(2, 3, new Add(sheet.getCell(2, 0), new Add(sheet.getCell(2, 1), sheet.getCell(2, 2))));
    sheet.setCell(1, 4, new Div(sheet.getCell(2, 3), new NumericConstant(0)));
    sheet.setCell(2, 4, new Append(sheet.getCell(2, 3), sheet.getCell(1, 4)));
    sheet.setCell(3, 4, new Append(sheet.getCell(2, 3), sheet.getCell(0, 3)));
    sheet.setCell(3, 5, new Add(sheet.getCell(0, 1), sheet.getCell(3, 4)));
    //sheet.setCell(1, 0, sheet.getCell(1, 3));
    System.out.println(sheet);
  }
}
\end{verbatim}
}

\noindent
ottenendo a video:

{\scriptsize
\begin{verbatim}
|          |        13|       -45|          |          |
|          |        11|        23|          |          |
|          |         1|         4|          |          |
|   totale:|        25|       -18|          |          |
|          |       ###|       ###|-18 totale|          |
|          |          |          |       ###|          |
\end{verbatim}
}

\noindent
dove \texttt{\#\#\#} indica un errore di valutazione.
Se si decommenta il penultimo comando del \texttt{main}, si crea un ciclo nel foglio elettronico
e la sua stampa risulta differente:

{\scriptsize
\begin{verbatim}
|          |       @@@|       -45|          |          |
|          |        11|        23|          |          |
|          |         1|         4|          |          |
|   totale:|       @@@|       -18|          |          |
|          |       ###|       ###|-18 totale|          |
|          |          |          |       ###|          |
\end{verbatim}
}

\begin{esercizio}
\textbf{[2 punti]}
Si definisca l'eccezione \texttt{EvaluationException} che estende \texttt{java.lang.RuntimeException}
ed indica che la valutazione di un'espressione \`e fallita per un errore di tipo o una divisione per zero.
Si definisca la sua sottoclasse \texttt{CyclicEvaluationException} che indica che la valutazione di un'espressione
\`e finita in un ciclo.
\end{esercizio}

\begin{esercizio}
\textbf{[3 punti]}
Si completi la seguente classe, che implementa un'espressione posta in una cella:

{\small
\begin{verbatim}
public abstract class Exp {
  public abstract int getNumericValue() throws EvaluationException;
  public abstract String getStringValue() throws EvaluationException;
  @Override public final String toString() { ... }
}
\end{verbatim}
}

\noindent
Il metodo \texttt{getNumericValue()} restituisce il valore dell'espressione se tale valore 
\`e numerico; lancia un'eccezione fra quelle dell'esercizio 1 altrimenti. Il metodo
\texttt{getStringValue()} restituisce il valore dell'espressione visto come stringa; questo \`e
possibile sempre, anche se l'espressione ha un valore intero, poich\'e un intero pu\`o essere convertito
nella stringa corrispondente. Il metodo \texttt{toString()} si comporta come un sinonimo di
\texttt{getStringValue()} ma non lancia eccezioni. Infatti, se la valutazione d\`a un errore, deve restituire
\texttt{\#\#\#} oppure \texttt{@@@}, a secondo della classe dell'eccezione.
\end{esercizio}

\begin{esercizio}
\textbf{[8 punti]}
Si completino le seguenti sottoclassi concrete di \texttt{Exp}, implementando quindi i suoi metodi
astratti. Non ci si preoccupi per ora di controllare se la valutazione di un'espressione
finisce in un ciclo.
%
{\small
\begin{verbatim}
public class NumericConstant extends Exp {  // costante numerica
  public NumericConstant(int value) { ... }
  {...}
}

public class StringConstant extends Exp {  // costante stringa
  public StringConstant(String value) { ... }
  {...}
}

public class Add extends Exp {  // addizione di due espressioni
  public Add(Exp left, Exp right) { ... }
  {...}
}

public class Div extends Exp {  // divisione fra due espressioni
  public Div(Exp left, Exp right) { ... }
  { ... attenzione al caso in cui il valore numerico di right e' zero! }
}

public class Append extends Exp {  // concatenazione fra due espressioni viste come stringa
  public Append(Exp left, Exp right) { ... }
  { ... }
}

public final class Cell extends Exp {  // un riferimento a una cella
  private Exp exp;
  protected Cell() { this.exp = new StringConstant(""); } // una cella nasce vuota
  public void setExp(Exp exp) { this.exp = exp; }
  { ... }
}

\end{verbatim}
}

\end{esercizio}

\begin{esercizio}
\textbf{[3 punti]}
Si completi la classe che implementa il foglio elettronico vero e proprio:

{\small
\begin{verbatim}
public class Sheet {
  private final Cell[][] cells;
  public Sheet(int width, int height) { ... inizializza cells a celle vuote }
  public void setCell(int x, int y, Exp exp) { ... setta exp come espressione della cella in x,y }
  public Cell getCell(int x, int y) { ... ritorna la cella in x,y }
  @Override public String toString() {
    String result = "";
    for (int y = 0; y < cells[0].length; y++) {
      result += "|";
      for (int x = 0; x < cells.length; x++) {
        String content = cells[x][y].toString();
        if (content.length() > 10) content = content.substring(0, 10);
        result += String.format("%10s|", content);
      }
      result += "\n";
    }
    return result;
  }
}
\end{verbatim}
}

\end{esercizio}

\begin{esercizio}
\textbf{[6 punti]}
Si modifichino la classe \texttt{Exp} e le sue sottoclassi concrete in modo da fare lanciare
una \texttt{CyclicEvaluationException} ai metodi \texttt{getNumericValue()} e
\texttt{getStringValue()} quando la valutazione finisce in un ciclo e quindi non pu\`o terminare.
\end{esercizio}

\end{document}
