\documentclass{article}[10pt]
\usepackage[pdftex]{graphicx}
\usepackage{amsfonts}
%****************enlarge layout
\textheight     243.5mm
\topmargin      -20.0mm
\textwidth      480pt
\hoffset        -80pt
%*****************theorems and such
\newcounter{esnu}
\newenvironment{esercizio}{\medskip \noindent {\bf Esercizio\addtocounter{esnu}{1} \arabic{esnu}}}{}
\pagestyle{empty}
\newcommand{\liff}{\mathrel{\leftrightarrow}}   % Logical IFF Symbol
\newcommand{\metaiff}{\Longleftrightarrow}      %iff in metatheory

\begin{document}

%\begin{tabular}{llclcr}
% \hspace{-35pt} &{\bf COGNOME:} & \hspace{100pt}        &{\bf NOME:}    & \hspace{100pt}        &{\bf MATRICOLA:}%\hspace{35pt} \\
%\hline
%\end{tabular}
\begin{center} {\bf Esame di Programmazione II, 15 settembre 2014}\end{center}
%\`

Una tokenizzazione \`e un iterabile su stringhe:

{\small\begin{verbatim}
public interface Tokenization extends Iterable<String> {
}
\end{verbatim}}

\noindent
L'idea \`e che, iterando su una tokenizzazione, si ottengano,
una alla volta, le sottostringhe di una scomposizione di una stringa di partenza
\texttt{s}, divisa sulla base di un criterio
specificato dalla specifica implementazione della tokenizzazione. Per esempio, una tokenizzazione
potrebbe usare un insieme di caratteri delimitatori, dopo i quali separare le sottostringhe di
\texttt{s}.

\begin{esercizio}
\textbf{[7 punti]}
%
Si scriva un'implementazione \texttt{StringTokenization} di \texttt{Tokenization}, che divide
una stringa \texttt{s}
subito dopo uno dei caratteri delimitatori forniti al momento della costruzione:

{\small
\begin{verbatim}
public class StringTokenization implements Tokenization {
  ...
  public StringTokenization(String s, String delimiters) { ... }

  @Override
  public Iterator<String> iterator() { ... }
}
\end{verbatim}}

\noindent
Per esempio, iterando su una \texttt{new StringTokenization("Questa\$e'\&una\&prova!", "\$\&")} si
devono ottenere una dopo l'altra le stringhe:

{\small
\begin{verbatim}
Questa$
e'&
una&
prova!
\end{verbatim}}

\noindent
Se la stringa dei delimitatori \`e vuota, il costruttore deve lanciare una
\texttt{NoDelimitersException}, che dovete scrivere.
\end{esercizio}

\begin{esercizio}
\textbf{[2 punti]}
%
Si scriva un'implementazione \texttt{CharacterTokenization} di \texttt{Tokenization}, che divide
la stringa \texttt{s}
subito dopo un ben preciso carattere \texttt{c} fornito al momento della costruzione:

{\small
\begin{verbatim}
public class CharacterTokenization extends StringTokenization {
  public CharacterTokenization(String s, char c) { ... }
}
\end{verbatim}}

\end{esercizio}

\begin{esercizio}
\textbf{[2 punti]}
%
Si scriva un'implementazione \texttt{SpaceTokenization} di \texttt{Tokenization}, che divide
la stringa \texttt{s} subito dopo uno spazio:

{\small
\begin{verbatim}
public class SpaceTokenization extends CharacterTokenization {
  public SpaceTokenization(String s) { ... }
}
\end{verbatim}}

\end{esercizio}

\begin{esercizio}
\textbf{[11 punti]}
%
Si scriva un'implementazione \texttt{DoubleTokenization} di \texttt{Tokenization}, che divide
la stringa \texttt{s} prima secondo un'altra tokenizzazione fornita al costruttore e poi
divide ciascuna di tali stringhe dopo uno dei caratteri delimitatori forniti al costruttore:

{\small
\begin{verbatim}
public class DoubleTokenization implements Tokenization {
  ...
  public DoubleTokenization(Tokenization base, String delimiters) { ... }

  @Override
  public Iterator<String> iterator() { ... }
}
\end{verbatim}}

\end{esercizio}

\newpage

\hrule

\vspace*{2ex}

Se tutto \`e corretto, un'esecuzione del seguente \texttt{main}:
%
{\small
\begin{verbatim}
public class Main {
  public static void main(String[] args) {
    Tokenization tok = new StringTokenization("Questa$e'&una&prova!", "$&");
    for (String s: tok)
      System.out.println(s);

    Tokenization tok2 = new DoubleTokenization(tok, "ou");
    for (String s: tok2)
      System.out.println(s);
  }
}
\end{verbatim}}

\noindent
dovr\`a stampare:

{\small
\begin{verbatim}
Questa$
e'&
una&
prova!
Qu
esta$
e'&
u
na&
pro
va!
\end{verbatim}}

\end{document}
