\documentclass{article}[10pt]
\usepackage[pdftex]{graphicx}
\usepackage{amsfonts}
\usepackage[italian]{babel}
%****************enlarge layout
\textheight     245.0mm
\topmargin      -22.0mm
\textwidth      480pt
\hoffset        -80pt
%*****************theorems and such
\newcounter{esnu}
\newenvironment{esercizio}{\medskip \noindent {\bf Esercizio\addtocounter{esnu}{1} \arabic{esnu}}}{}
\pagestyle{empty}
\newcommand{\liff}{\mathrel{\leftrightarrow}}   % Logical IFF Symbol
\newcommand{\metaiff}{\Longleftrightarrow}      %iff in metatheory

\begin{document}

%\begin{tabular}{llclcr}
% \hspace{-35pt} &{\bf COGNOME:} & \hspace{100pt}        &{\bf NOME:}    & \hspace{100pt}        &{\bf MATRICOLA:}%\hspace{35pt} \\
%\hline
%\end{tabular}
\begin{center} {\bf Semplificazione dell'esame di Programmazione II del 6 luglio 2012}\end{center}

Si consideri il seguente programma:
%
{\small
\begin{verbatim}
public class Main {
  public static void main(String[] args) throws DuplicatedSampleException, InconsistentSampleSizeException {
    Sample s1 = new Sample("Cars", new float[] { 187.2f, 201.4f, 88.2f, 75.8f, 156.1f } );
    Sample s2 = new Sample("Bikes", new float[] { 91.3f, 98.7f, 111.3f, 120.4f, 100.2f } );		
    Sample s3 = new Sample("Motorbikes", new float[] { 122.3f, 118.7f, 144.0f, 125.4f, 88.9f } );		
    Sample s4 = new Sample("Skateboards", new float[] { 21.3f, 6.3f, 44.0f, 24.3f, 18.9f } );		
    Plot p1 = new SequentialPlot();
    p1.add(s1); p1.add(s2); p1.add(s3); p1.add(s4);

    Sample s5 = new Sample("Cars", new float[] { 187.2f, 201.4f, 88.2f, 75.8f, 156.1f } );
    Sample s6 = new Sample("Bikes", new float[] { 91.3f, 98.7f, 111.3f, 120.4f, 100.2f } );		
    Sample s7 = new Sample("Motorbikes", new float[] { 122.3f, 118.7f, 144.0f, 125.4f, 88.9f } );		
    Plot p2 = new AlternatePlot();
    p2.add(s5); p2.add(s6); p2.add(s7); p2.add(s4);

    System.out.println(p1); System.out.println(p2); System.out.println(p1.equals(p2));
  }
}
\end{verbatim}
}

\noindent
che alla fine del compito dovr\`a stampare:
%
{\scriptsize
\begin{verbatim}
Cars:
| ************************************* (187.2)
| **************************************** (201.4)
| ***************** (88.2)
| *************** (75.8)
| ******************************* (156.1)

Bikes:
| @@@@@@@@@@@@@@@@@@ (91.3)
| @@@@@@@@@@@@@@@@@@@ (98.7)
| @@@@@@@@@@@@@@@@@@@@@@ (111.3)
| @@@@@@@@@@@@@@@@@@@@@@@ (120.4)
| @@@@@@@@@@@@@@@@@@@ (100.2)

Motorbikes:
| $$$$$$$$$$$$$$$$$$$$$$$$ (122.3)
| $$$$$$$$$$$$$$$$$$$$$$$ (118.7)
| $$$$$$$$$$$$$$$$$$$$$$$$$$$$ (144.0)
| $$$$$$$$$$$$$$$$$$$$$$$$ (125.4)
| $$$$$$$$$$$$$$$$$ (88.9)

Skateboards:
| **** (21.3)
| * (6.3)
| ******** (44.0)
| **** (24.3)
| *** (18.9)


           Cars| ************************************* (187.2)
          Bikes| @@@@@@@@@@@@@@@@@@ (91.3)
     Motorbikes| $$$$$$$$$$$$$$$$$$$$$$$$ (122.3)
    Skateboards| **** (21.3)

           Cars| **************************************** (201.4)
          Bikes| @@@@@@@@@@@@@@@@@@@ (98.7)
     Motorbikes| $$$$$$$$$$$$$$$$$$$$$$$ (118.7)
    Skateboards| * (6.3)

           Cars| ***************** (88.2)
          Bikes| @@@@@@@@@@@@@@@@@@@@@@ (111.3)
     Motorbikes| $$$$$$$$$$$$$$$$$$$$$$$$$$$$ (144.0)
    Skateboards| ******** (44.0)

           Cars| *************** (75.8)
          Bikes| @@@@@@@@@@@@@@@@@@@@@@@ (120.4)
     Motorbikes| $$$$$$$$$$$$$$$$$$$$$$$$ (125.4)
    Skateboards| **** (24.3)


true
\end{verbatim}
}

\begin{esercizio}
\textbf{[5 punti]}
Una \texttt{Sample} \`e una sequenza di valori con un nome associato. Si completi la sua classe:

{\small
\begin{verbatim}
public class Sample {
  private final String name;
  private final float[] values;

  public Sample(String name, float[] values) {
    this.name = name;
    this.values = values;
    ....
  }

  public String getName() { return name; }
  public int getSize() { return values.length; }
  public float getMax() { .... }
  public float getValue(int pos) { return values[pos]; }
  public boolean equals(Object other) ....
}
\end{verbatim}}

\noindent
in modo che \texttt{getMax()} ritorni il massimo valore nella sample e che
\texttt{equals} consideri uguali due \texttt{Sample} se e solo se hanno stesso nome e stesso
numero di valori nello stesso ordine. Il costruttore deve lanciare una
\texttt{java.lang.IllegalArgumentException} se uno o pi\`u dei valori \`e negativo.
\end{esercizio}

\begin{esercizio}
\textbf{[7 punti]}
Un \texttt{Plot} \`e una sequenza di un numero arbitrario di \texttt{Sample}, tutte con nomi diversi
fra di loro. \`E possibile
aggiungere una \texttt{Sample} a un \texttt{Plot} ed \`e possibile aggiungere a un \texttt{Plot} tutte le
\texttt{Sample} di un altro \texttt{Plot}. Si completi la sua classe:

{\small
\begin{verbatim}
public abstract class Plot {
  private Sample[] samples;

  protected Plot() { .... }
  public final void add(Sample sample) throws DuplicatedSampleException, InconsistentSampleSizeException....
  public final void add(Plot plot) throws DuplicatedSampleException, InconsistentSampleSizeException....

  public final float getMax() {  // il massimo valore fra tutte le sample
    float max = 0f;
    for (Sample sample: samples)
      max = Math.max(max, sample.getMax());
    return max;
  }

  public final boolean equals(Object other) ....
  protected final Sample[] getSamples() { return samples; }
  public abstract String toString();
}
\end{verbatim}}

\noindent
I metodi \texttt{add} devono lanciare una \texttt{DuplicatedSampleException} se si prova ad aggiungere
una \texttt{Sample} che ha lo stesso nome di una delle \texttt{Sample} gi\`a presente nel \texttt{Plot};
devono lanciare una \texttt{InconsistentSampleSizeException} se si prova ad aggiungere una \texttt{Sample}
che ha una lunghezza (\texttt{getSize}) diversa dalla lunghezza delle altre \texttt{Sample} gi\`a presenti
nel \texttt{Plot}. Le due classi di eccezione devono essere definite.
Due \texttt{Plot} devono essere \texttt{equals} se e solo se contengono lo stesso numero di \texttt{Sample}
e queste ultime sono \texttt{equals} a due e due e nello stesso ordine (si
studi l'esempio della prima pagina).
\end{esercizio}

\begin{esercizio}
\textbf{[10 punti]}
Si scrivano le classi \texttt{SequentialPlot} e \texttt{AlternatePlot} che estendono
\texttt{Plot} ridefinendo il metodo astratto \texttt{toString}. Tali ridefinizioni devono restituire
delle stringhe come nell'esempio della pagina precedente: in \texttt{SequentialPlot}
viene presentata una \texttt{Sample} alla volta; in \texttt{AlternatePlot} le \texttt{Sample}
sono alternate. In entrambi i casi, i caratteri alternati sono \texttt{*@\$}. La lunghezza delle
barre orizzontali deve essere proporzionale al valore che ciascuna barra
rappresenta. Si riservino 40 caratteri per il valore pi\`u grande (metodo \texttt{getMax}).
\end{esercizio}

\end{document}
